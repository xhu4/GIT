%        File: ALAHW1.tex
%     Created: Sun Sep 20 08:00 PM 2015 MDT
% Last Change: Sun Sep 20 08:00 PM 2015 MDT
%
\documentclass[12pt]{article}

%%%%%%%%%%%%%%%%%%%%%%%%%%%%%%%%%%%%%%%%%%%%%%%%%%%%%%

\usepackage{euscript,fullpage,amssymb}
\usepackage{amsmath}
\usepackage{mathtools}
\usepackage{amsfonts}
\usepackage[shortlabels]{enumitem}
\usepackage{amsthm}
\usepackage{siunitx}
%\usepackage{listings}
%\usepackage[bw,framed]{mcode}
%\usepackage[letterpaper]{geometry}
%\usepackage{graphicx}
%\usepackage{float}
%\geometry{
%  inner=26mm,
%}
%%%%%%%%%%%%%%%%%%%%%%%%%%%%%%%%%%%%%%%%%%%%%%%%%%%%%%%%%%%
\newcommand{\Q}{{\mathbb{Q}}}
\newcommand{\sr}[1]{\sqrt{#1}}
\newcommand{\vv}{\mathbf{v}}
\newcommand{\ww}{\mathbf{w}}
\newcommand{\0}{\mathbf{0}}
\newcommand{\F}{\mathbb{F}}
\newcommand{\R}{\mathbb{R}}
\newcommand{\N}{\mathbb{N}}
\newcommand{\C}{\mathbb{C}}
\newcommand\PP{\mathcal{P}}
\newcommand\f{\frac}
\newcommand\Nul{\mathrm{null}}
\newcommand\Ran{\mathrm{range}}
\newcommand\id{\mathrm{id}}
\newcommand{\ids}[1]{\mathrm{id}_#1}

\newcommand{\gp}[1]{\langle #1 \rangle}

\newcommand\sse\subseteq
\newcommand\sub\subset
\newcommand\ssm\smallsetminus
\renewcommand\L{\mathcal{L}}
\newcommand{\pf}{\underline{\textit{Proof.}}}
\newcommand{\fahrenheit}{\SIUnitSymbolDegree F}
\renewcommand{\F}[1]{\SI{#1}{\fahrenheit}}
%%%%%%%%%%%%%%%%%%%%%%%%%%%%%%%%%%%%%%%%%%%%%%%%%%%%%%%%%%%%%%%%

\usepackage{framed}

\begin{document}

%%%%%%%%%%%%%%%%%%%%%%%%%%%%%%%%%%%%%%%%%%%%%%%%%%%%%%%%%%%%%%%%

\setlength{\parskip}{0.125in}

%\newcounter{problem}
%\newcommand{\prob}{\noindent\stepcounter{problem}\arabic{problem}.~~}

%\newcounter{subproblem}[problem]
%\newcommand{\subprob}{\stepcounter{subproblem}(\alph{subproblem})\quad }

%%%%%%%%%%%%%%%%%%%%%%%%%%%%%%%%%%%%%%%%%%%%%%%%%%%%%%%%%%%%%%%%

\centerline{\bf \Large MATH 5490 --- HPC-xTC}

\bigskip
\centerline{\large Home Work \#2 \qquad \textbf{Group 5} \textsc{Jorge Flores, Xiukun Hu, Grigorii Sarnitskii }}

\section{Results}

The code is capable with square blocks (i.e.~with \texttt{blk\_rows} and \texttt{blk\_cols} to be the same) and multiple blocks in A and B.
The multiplication result will be written to disk in file(s) \texttt{C.*.*}, and can be printed on screen.

The result is examed by \textsc{Matlab} for $100\times100$, $1000\times1000$ and $5000\times5000$ block multiplication. The absolute difference between two results is less than $1e-11$ for all tests.

Total time spent for $5000\times5000$ matrices multiplication for this code is $1.37e3$ seconds and computing time is $1.36e3$ seconds, while for \textsc{Matlab} the computing time is only $4.28$ seconds.


\section{Requirement}

Unzip the folder and every file needed is inside. Use \texttt{make run} command to run it.

\section{Compute Bound}

For MacBook Pro with 2.7GHz Intel Core i5, 8G DDR3 memory, it gets compute bound when the block is over $50\times 50$.

For Lenovo with 2.4GHz Intel Core i7, running in a 64bit Red Hat virtual machine, it gets compute bound when the block is over $50\times 50$.


\end{document}

